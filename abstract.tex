% $Log: abstract.tex,v $
% Revision 1.1  93/05/14  14:56:25  starflt
% Initial revision
%
% Revision 1.1  90/05/04  10:41:01  lwvanels
% Initial revision
%
%
%% The text of your abstract and nothing else (other than comments) goes here.
%% It will be single-spaced and the rest of the text that is supposed to go on
%% the abstract page will be generated by the abstractpage environment.  This
%% file should be \input (not \include 'd) from cover.tex.

\deleted{The goal of fusion energy research is to build a profitable reactor. This thesis develops a cost estimate model for fusion reactors from a physicist's perspective. It then applies it to the two main modes of operation for a tokamak reactor: pulsed and steady-state. In the end, an apples-to-apples comparison is developed, which is used to explain: the relative advantages of pulsed and steady-state operation, as well as, the design parameters that provide the most leverage in lowering machine costs. The most notable of these is the magnetic field strength -- which should be doubled by ongoing research efforts at MIT using high-temperature superconducting (HTS) tape.}

\added{The goal of fusion energy research is to build an economically competitive reactor. This is difficult due to the complicated system composing a reactor and the nonlinearities it entails. Practically, to even get to the neighborhood of an economic reactor requires hundreds of simulations -- which in turn necessitate quick running fusion systems codes. Moving towards these economic reactors then involves finding what design parameters provide the most leverage in lowering reactor costs.}

\added{As highlighted by the difference between European and American designs, however, the most important decision for tokamaks is whether to run them as \emph{pulsed} or \emph{steady-state}. This paper aims to fairly compare the two modes of operation using a single, comprehensive model. Benchmarked against other codes, this model actually shows that no fusion reactor is achievable without some technological advancements. This can be seen through every referenced design using nonstandard values of $H$ and $N_G$.}

\added{The interesting result this paper shows is that developing high-temperature superconducting (HTS) tape could actually make both steady-state and pulsed tokamaks economically competitive against solar and coal. Further, this HTS tape actually has different best uses for the two modes of operation, appearing in the magnet structures of: TF coils for steady state and the central solenoid for pulsed. Developments in this technology should produce economic reactors within the coming decade.}
