\chapter{Concluding Reactor Discussion}

The goal of this document was to develop a simple fusion systems model that can work for both pulsed and steady-state tokamaks. The main conclusion was that the best way to build a more efficient, compact reactor is to invest in strong magnets -- as MIT is doing with high-temperature superconducting (HTS) tape. Further it was shown that to best utilize materials, the tape should be incorporated into the toroidal field coils for steady-state machines and in the central solenoid for pulsed ones.

Although some skepticism should be allotted to these conclusions, it was shown that this simple algebraic solver matched sophisticated multiyear research studies with speed and ease. This model may not provide an engineer's rigor in measuring cost, but the same can be said for any code or theory. The fusion system is as nonlinear a problem as they come, but we still managed to build a framework that can hone even a well-trained physicist's intuition.

The final point to make is that this model actually predicts that HTS technology can provide the optimum magnetic field strength for a reactor. Once HTS doubles the maximum achievable teslas, the law of diminishing returns heavily kicks in. This of course assumes H-Mode D-T plasmas at the Greenwald density limit.